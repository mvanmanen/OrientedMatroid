\documentclass[12pt,oneside]{amsart}
\usepackage{latexsym}
\usepackage{amssymb}
\usepackage{color}
\usepackage{graphicx}
\usepackage[all]{xy}
\newtheorem{theorem}{{\sc Theorem}}
\newtheorem{definition}{{\sc Definition}}
\newtheorem{lemma}{{\sc Lemma}}
\newtheorem{corollary}{{\sc Corollary}}
\newtheorem{proposition}{{\sc Proposition}}
\newtheorem{remark}{{\sc Remark}}
\newtheorem{example}{{\sc Example}}
\newcommand{\ZZ}{{\mathbb Z}}
\newcommand{\RR}{{\mathbb R}}
\newcommand{\CC}{{\mathbb C}}
\newcommand{\Tau}{\mathcal{C}}
\newcommand{\calA}{{\mathcal A}}
\newcommand{\Discrete}[1]{\mathfrak{#1}}
\def\colP#1#2{\{P_{#1},\cdots, P_{#2}\}}
\def\colQ#1#2{\{Q_{#1},\cdots, Q_{#2}\}}
\def\dP#1#2{d(P_{#1}, P_{#2})}
\def\sP#1#2{S(P_{#1}, P_{#2})}
\def\tP#1#2#3{\Delta(P_{#1}, P_{#2}, P_{#3})}
\def\P#1#2{\{P_{#1}, P_{#2}\}}
\def\PP#1#2#3{\{P_{#1}, P_{#2}, P_{#3}\}}
\def\PPP#1#2#3#4{\{P_{#1}, P_{#2}, P_{#3}, P_{#4}\}}
\setlength{\hoffset}{-1in}
\setlength{\voffset}{-1in}
\setlength{\textwidth}{7in}
\setlength{\textheight}{10in}
\setlength{\mathsurround}{1mm}
\DeclareMathOperator{\diffd}{d}
\DeclareMathOperator{\rk}{rk}
%\DeclareMathOperator{\Im}{Im}
\DeclareMathOperator{\tr}{tr}
\DeclareMathOperator{\xsup}{sup}
\DeclareMathOperator{\CH}{CH}
\DeclareMathOperator{\Up}{Up}
\DeclareMathOperator{\Down}{Down}
\DeclareMathOperator{\Dual}{Dual}
\DeclareMathOperator{\Bary}{Bary}
\DeclareMathOperator{\Pow}{Pow}
\DeclareMathOperator{\Del}{Del}
\DeclareMathOperator{\Sep}{Sep}
\DeclareMathOperator{\Act}{Act}
\DeclareMathOperator{\Aff}{Aff}
\DeclareMathOperator{\Cone}{Cone}
\DeclareMathOperator{\Star}{Star}
\DeclareMathOperator{\Link}{Link}
\DeclareMathOperator{\Relint}{Relint}
\DeclareMathOperator{\Md}{Md}
\newcommand{\trev}[1]{\lvert #1 \rvert}
\newcommand{\treV}[1]{\lVert #1 \rVert^2}
\def\prc#1{{\prec_{#1}}}
\def\scc#1{{\succ_{#1}}}
\newcommand{\Upo}{\Up_1}
\newcommand{\Downo}{\Up_1}
\newcommand{\mcA}{\mathcal{A}}
\newcommand{\mcB}{\mathcal{B}}
\newcommand{\mcD}{\mathcal{D}}
\newcommand{\mcCa}{{{\mathcal C}^\ast}} % the cocircuits
\newcommand{\mcCap}{{{\mathcal C}^{\prime\ast}}}
\newcommand{\mcC}{{\mathcal C}} % the circuits
\newcommand{\mcCp}{{{\mathcal C}^\prime}}
\newcommand{\mcL}{{\mathcal L}} % the covectors
\newcommand{\mcLp}{{{\mathcal L}^\prime}}
\newcommand{\mcV}{{\mathcal V}} % the vectors
\newcommand{\mcVp}{{{\mathcal V}^\prime}}
\newcommand{\mcM}{{\mathcal M}}
\newcommand{\mcMp}{{{\mathcal M}^\prime}} % the underlying set
\newcommand{\umcM}{{\underline{\mathcal M}}}
\newcommand{\umcMp}{{\underline{\mathcal M}^\prime}} % the underlying set
\newcommand{\mcN}{{\mathcal N}} % jet bundle
\newcommand{\mcNp}{{{\mathcal N}^\prime}} % the underlying set
\newcommand{\umcN}{{\underline{\mathcal N}}}
\newcommand{\umcNp}{{\underline{\mathcal N}^\prime}} % the underlying set
\begin{document}
\section{Computation of the center}
We have that $P_1,\cdots , P_d$ are $d$ points in $\RR^n$.
We want to compute the center.
We need $d\leq n+1$.
Set $y=f_i(x)=\langle x,P_i \rangle - \frac12\lVert P_i \rVert^2+\frac12 r_i^2$ where $r_i$ is the radius of the circle around $P_i$.
Set $c_i=\frac12 r_i^2-\frac \lVert P_i \rVert^2$. 
For the point $Q$ to be the center of the circle the functions $f_i(Q)$ all need to be equal and the function 
$f(x)=\max_if_i(x)$ needs to be at the maximum.
That means that $g(x)=\frac12\lVert x\rVert^2-f(x)$ needs to be a critical point.
So that means that $0$ is in the convex hull of the $d$ vectors $x-P_i$, because the gradient of each individual $g_i(x)$ is $x-P_i$.
So the center needs to be in $\CH\left(P_1,\cdots ,P_d\right)$.
Hence to determine $x$ we have $d-1$ unknowns that are fixed by the $d-1$ equalities $f_i(x)=f_d(x)$, for $i=1,\cdots , d-1$.
Now note that the setup is invariant under translations.
So we can assume $P_d=0$. Then $f_d(x)=\frac12r_d^2$.
Then the center $q=\sum_{i=1}^{d-1}\lambda_iP_i$ with $0\leq\lambda_i$ for $i=1,\cdots ,d-1$ and $\sum_{i=1}^{d-1}\leq 1$.
Moreover $f_i(q)=f_d(q)=\frac12r_d^2$. 
Hence we have two linear equations combined to get $q$, or the $\lambda$ vector.
We need to compute the $(d-1)\times(d-1)$ matrix with entries $a_{i,j}=\langle P_i,P_j\rangle$, which is invertible as long as the vectors 
span a full volume, i.e.\ they are linearly independent. If they are not then we need to reconsider the problem in the subspace spanned by the origin 
and the points $P_i$ for $i=1,\cdots,d-1$. 
Let $\mathbb{P}$ be the $(d-1)\times n$ matrix spanned by the points $P_1$ to $P_{d-1}$. The matrix
$\mathbf{P}\mathbf{P}^T$ is the matrix with entries $\langle P_i,P_j\rangle$.
To compute the Cholesky decomposition we would like to use that we already have the decomposition in $\mathbf{P}$ and $mat
Then we need to have $\mathbf{P}\mathbf{P}^T\lambda=b$ where $b$ is the vector with entries $\frac12\left(r_d^2-\lVert P_i \rVert^2 -r_i^2\right)$ for 
$i=1, \cdots ,d-1$.

Then invert $A$ and the solution for $$.
Use \begin{verbatim} cho_solve $\end{verbatim}
We can use 

\section{}
We want to show that with the functions $f_i$ we can easily 
\end{document}
